\documentclass[
	letterpaper, % Paper size, specify a4paper (A4) or letterpaper (US letter)
	10pt, % Default font size, specify 10pt, 11pt or 12pt
]{CSUniSchoolLabReport}

\usepackage{fancyvrb}
\usepackage{multicol}
\usepackage{subcaption}

\captionsetup[subfigure]{labelformat=empty}

\title{Analyzing radioactive decay of multiple samples with different half lives.}

\author{Sebastien \textsc{Psarianos}\\ Sofiya \textsc{P'yavka}}

\date{\today}

\begin{document}

\maketitle

\begin{center}
	\begin{tabular}{l r}
		Date Performed: & September 29, 2022 \\
	\end{tabular}
\end{center}

\vspace{20pt}
\section{Methods and Procedure}
\vspace{20pt}
\textbf{Background Radiation} A Geiger counter was set up in the laboratory with no radioactive sample present. Particle count measurements were taken for every $20$ second interval and this was repeated for $1200$ seconds ($60\times20$ second intervals). The data collected in this portion of the laboratory was used as a baseline measurement for the background radiation in the laboratory.\\

\textbf{Experiment 1} (Barium Sample): A sample of Barium-137 was then placed near the Geiger counter. Particle count measurements were taken again over the same intervals as the background radiation measurements ($60\times20$ second intervals).\\

\textbf{Experiment 2} (Fiesta Plate Sample): A fiesta plate with a coating that contains uranium was then placed near the Geiger counter. The same particle count measurements were again taken for $1200$ seconds, however for this experiment, $3$ second intervals were used ($400\times3$ second intervals).
\vspace{20pt}
\section{Results}
\vspace{20pt}
Note: All curve fitting regression values were calculated using the \lstinline{curve_fit} function from the \lstinline{scipy.optimize}. Referenced functions, equations and calculations are detailed in the \textbf{Appendix} section in addition to the raw data and calulated uncertainties.
\newpage
{\Large\textbf{Experiment 1}}
\begin{figure}[H]
	\begin{subfigure}{0.45\textwidth}
		\includegraphics[width=\textwidth]{linearBariumGraph}
		\caption{\textbf{Figure 1: Particle counts over 20 second interval with mean background radiation subtracted vs time for a sample of Barium-137. Linear, exponential and theoretical half life model regressions are included. Residuals for the linear and exponential models are included.}}
	\end{subfigure}
	\quad
	\begin{subfigure}{0.45\textwidth}
		\includegraphics[width=\textwidth]{logBariumGraph}
		\caption{\textbf{Figure 2 Linearized particle counts over 20 second interval with mean background radiation subtracted vs time for a sample of Barium-137.  Linear, exponential and theoretical half life model regressions are included. Residuals for the linear and exponential models are included.}}
	\end{subfigure}
\end{figure}
All plotted values have had the mean background radiation measured in the laboratory subtracted from
them (approximately $3.42$). One measurement for the interval $1040-1060$ was removed from graphs since
it is outside the log domain. This is discussed further in the analysis section.\\\\
{\large\textbf{Curve fitting}}\\
Three \lstinline{curve_fit} regressions were performed on the data using an exponential model, a linear model and a
model based on the theoretical half life of Barium-137. The derivation of the theoretical model is detailed in
\textbf{Calculation 1}. \textbf{Equation 1}, \textbf{Equation 2} and \textbf{Equation 3} were used for the linear, exponential and
theoretical models respectively. The implementations shown in \textbf{Function 1}, \textbf{Function 2} and \textbf{Function 3} were used for \lstinline{curve_fit}. The exponential
and theoretical model regressions were performed on the raw data set and the linear regression was
performed using the linearized data.\\\\
All data linearization was done using a natural logarithm on the corresponding y-axis. This was used for
linear modelling in addition to the linearized plotting in \textbf{Figure 2} of the data, exponential and theoretical
models. To plot the linear model on \textbf{Figure 1}, the output values of the linear model regression were
de-linearized by taking the exponential of them (base $e$).\\\\
{\large\textbf{Uncertainty Calculations}}\\
Uncertainty in count measurements were all calculated using \textbf{Equation 8}. This was done programmatically
for all measured values using the python implementation \textbf{Function 5}. Sample calculations for the first reading
are shown in \textbf{Calculation 2}.\\\\
Uncertainty was propagated for the linearization by using the logarithmic error propagation shown in
\textbf{Equation 6}. This again was done programmatically for all values using the python implementation \textbf{Function 4}.
Sample calculations for the first reading are shown in \textbf{Calculation 3}\\\\
{\Large\textbf{Experiment 2}}
\begin{figure}[H]
	\begin{subfigure}{0.45\textwidth}
		\includegraphics[width=\textwidth]{fiestaDistributionGraph}
		\caption{\textbf{Figure 3: Probability density for various ranges of counts measured over 3s intervals. Data is from the fiesta plate sample measurements. Includes both Poisson and Gaussian distributions. }}
	\end{subfigure}
	\quad
	\begin{subfigure}{0.45\textwidth}
		\includegraphics[width=\textwidth]{backgroundDistributionGraph}
		\caption{\textbf{Figure 4: Probability density for various ranges of counts measured over 20s intervals. Data is from the background radiation sample measurements. Includes both Poisson and Gaussian distributions. }}
	\end{subfigure}
\end{figure}
{\large\textbf{Histograms}}\\
Histogram plots in \textbf{Figure 3} and \textbf{Figure 4} were done using the \lstinline{hst} function from the \lstinline{matplotlib.pyplot}
package with the \lstinline{density} set to true, generating a probability density histogram rather than a count
histogram. All binning was done automatically by the \lstinline{hst} function. The plot in \textbf{Figure 3} shows the count
values after the mean background radiation had been subtracted.\\\\
{\large\textbf{Gaussian and Poisson Distributions}}\\
For the Poisson distributions, the \lstinline{poisson.pmf} function from \lstinline{scipy.stats} was used. This function is an implementation of the Poisson probability mass function (\textbf{Equation 9}). For both the background and fiesta plate datasets, the provided $\mu$ value was the mean radiation that was detected over the course of the respective experiment.\\

For the Gaussian distributions, the \lstinline{norm.pdf} function from \lstinline{scipy.stats} was used. This function uses the probability density function (\textbf{Equation 10}) to generate a normal distribution. The scale and location for each distribution was set based on the average count over the respective experiment ($\mu$). One standard deviation ($\sigma$) was set to $\sqrt\mu$ and the location of the distribution was set to $\mu$.
\newpage
\section{Analysis}
\vspace{20pt}
{\Large\textbf{Experiment 1}}\\

The chi-squared values for each graph were calculated programmatically using \textbf{Function 6} (implementing
\textbf{Equation 7}). Details of how the half life and initial intensity were calculated are shown in \textbf{Calculation 4}.
The calculated half life and initial intensity in addition to the chi-squared value for the linear and exponential are as follows.\\

\textbf{Linear Regression}: The half-life was determined to be $145 \pm 6s$ and the initial intensity value ($I_0$) was
determined to be $760\pm70\frac{J}{m^2 s}$. The chi-squared value was calculated as $\chi^2 = 1.30$.\\

\textbf{Exponential Regression}: The half life was determined to be $148 \pm 3s$ and the initial intensity value ($I_0$) was
determined to be $717\pm8\frac{J}{m^2s}$. The chi-squared value was calculated $\chi^2 = 0.98$.\\

The exponential model gave a value closer to the theoretical half life of 156 seconds, however, the
theoretical half life does not fall in the uncertainty range of either model.\\
In \textbf{Figure 1} and \textbf{Figure 2}, both the linear and exponential curves visually fit the experimental data quite well,
with the exponential regression method tending closer to the theoretical curve. This can be clearly seen in \textbf{Figure 2}, where the exponential model is visually closer to the theoretical curve.\\
Analyzing the chi-squared values for each graph, the nonlinear regression's value of $0.98$ is much closer to
the ideal value of $1$ than the linear regression's value of $1.30$. Although the value determined from the linear
model deviates slightly farther, it is still relatively close to the accepted value. Ultimately, the reduced chi-squared
values indicate that both models are good fits. However, nonlinear regression provides values that more closely approximate the experimental data.\\

{\Large\textbf{Experiment 2}}\\

Examining the fiesta plate data in \textbf{Figure 3}, the Poisson distribution is a very close approximation of the
Gaussian distribution. There was a larger visual discrepancy between the two distributions in the background
data in \textbf{Figure 4}. The Poisson distribution in \textbf{Figure 4} is notably shifted left of the Gaussian
distribution. Due to the discrete, non-negative nature of Poisson distributions, as $\mu\rightarrow 0$, they will become less and less
symmetrical and will fall off much more steeply on the left. This is not reflected in Gaussian
distributions which have the possibility of having positive probabilities for negative numbers. The Poisson
distribution in \textbf{Figure 4} is clearly not symmetrical and has a noticeably steeper slope on the left side. This
indicates why \textbf{Figure 3} with its much higher $\mu$ value is a much better approximation of a Gaussian distribution.\\

Visually, the experimental data in \textbf{Figure 3} appears to be a much better approximation of both the Poisson
and Gaussian distributions. The experimental data in \textbf{Figure 4} seems to be a better approximation of the
Poisson distribution which makes sense due to the positive discrete data set provided by a Geiger counter
and the data's proximity to counts of $0$.
\newpage
\vspace{20pt}
\section{Discussion}
\vspace{20pt}
{\Large\textbf{Experiment 1}}\\

Although the experimental half-lives were slightly less than the expected value, this was likely due to the
background radiation which had to be subtracted from the collected data. The experiment ultimately
verifies that the intensity of radiation decays exponentially and agrees that the half-life of Barium-137 is
approximately 2.6 minutes. If the mean background radiation could be calculated with a larger sample size, a more
accurate background radiation level could likely be determined. Modifications to a future experiment could
include a longer period of background radiation measurements with a smaller interval to get more data points
and by extension a more accurate approximation.\\

{\Large\textbf{Experiment 2}}\\

More count data points would likely create a more accurate model of the radioactive emission for both
the background and the fiesta plate. This could be achieved by measuring both for a longer period of time.\\

To get a Gaussian and Poisson distribution that are more similar, altering the measurement method to get
a higher $\mu$ value would be the solution. This is due to the fact that Poisson distributions tend towards Gaussian
normal distributions as $\mu\rightarrow \infty$. This could be achieved experimentally by extending the length of the
intervals to get a larger number for each count value. This would likely make a significant difference for the
background data as it is so close to $0$.\\\\

Experiment 1 and experiment 2 can both be improved by altering the intervals for the background measurement. Improvements would come from a background measurement interval length decrease in experiment 1 and an increase in experiment 2.
\vspace{20pt}
\section{Conclusion}
\vspace{20pt}
The results of this lab for both experiments 1 and 2 were generally quite close to their theoretical values.
However, the experimental values and models both deviated away from expected values. Thus, it is evident that a factor, such as the background radiation, slightly skewed the experimental results. This is likely an issue with the background radiation measurement. For a future experiment, it would be advantageous to record the background radiation twice over different intervals. This would give two datasets that provide more useful information for each experiments' analysis.
Utilizing multiple Geiger counters to get a larger data set for both the sample and background measurements could also be beneficial as it would increase the likelihood that detection events occur for each radiation emission.
\vspace{20pt}
\section{Appendix}
\vspace{20pt}
{\Large\textbf{Equations}}\\
\begin{tabular}{p{0.45\linewidth} p{0.45\linewidth}}
$$f(x) = ax+b$$
\begin{center}
	\textbf{Equation 1: Linear Model}
\end{center}
&
$$f(x) = be^{ax}$$
\begin{center}
	\textbf{Equation 2: Exponential Model}
\end{center}\\

$$I(t)= I_0 e^{-\frac{t \ln{2}}{156}}$$
\begin{center}
	\textbf{Equation 3: Theoretical Model}
\end{center}
&
$$I(t) = I_0e^{-\frac{t}{\tau}}$$
\begin{center}
	\textbf{Equation 4: Mean isotope lifetime equation}\\
\end{center}\\

$$\tau = \frac{t_{1/2}}{\ln{2}}$$
\begin{center}
	\textbf{Equation 5: Mean isotope lifetime to half life conversion}
\end{center}
&
$$ u\left(\ln(x_i)\right) = \pm\left|\frac{u(x_i)}{x_i}\right|$$
\begin{center}
	\textbf{Equation 6: Error Propagation for logarithms}
\end{center}\\

$$\chi^2 = \sum_{i=1}^N\left(\frac{y_i-y(x_i)}{u(y_i)}\right)$$
\begin{center}
	\textbf{Equation 7: Chi-Squared Metric}
\end{center}
&
$$u(N_i) = \pm\sqrt{N_{total, i} + \bar{N}_{b}}$$
\begin{center}
	\textbf{Equation 8: Geiger Counter Uncertainty}
\end{center}\\
$$P_\mu(n) = e^{-\mu} \frac{\mu^n}{\Gamma(n+1)}$$
\begin{center}
	\textbf{Equation 9: Poisson mass distribution function}
\end{center}
&
$$f(x) = \frac{e^{-x^2/2}}{\sqrt{2\pi}}$$
\begin{center}
	\textbf{Equation 10: Gaussian probability density function}
\end{center}
\end{tabular}
\vspace{20pt}\\
{\Large\textbf{Python Functions}}
\vspace{20pt}\\
{\large\textbf{Models}}
\begin{verbatim}
def linear_model(values, a, b) -> any:
     return a * values + b
\end{verbatim}
\begin{center}
	\textbf{Function 1: Linear Model (implements Equation 1)}
\end{center}
\vspace{5pt}
\begin{verbatim}
def exponential_model(values, a, b) -> any:
     return b * np.exp(a * values)
\end{verbatim}
\begin{center}
	\textbf{Function 2: Exponential Model (implements Equation 2)}
\end{center}
\vspace{5pt}
\begin{verbatim}
def theoretical_model(values, b) -> any:
     return b * np.exp((-1 / 156 * np.log(2)) * values)
\end{verbatim}
\begin{center}
	\textbf{Function 3: Theoretical Model (implements Equation 3)}
\end{center}
\vspace{10pt}
{\large\textbf{Uncertainty}}
\begin{verbatim}
def logarithmic_error_propagation(value: any, uncertainty: any) -> float:
     """Return the propogated error for the logarithm of a value"""
     return abs(uncertainty / value)
\end{verbatim}
\begin{center}
	\textbf{Function 4: Logarithmic Error Propagation (implements Equation 6)}
\end{center}
\vspace{10pt}
\begin{verbatim}
def calculate_uncertainty(count, mean_background) -> any:
     """Return the uncertainty of the sample.
     """
     return np.sqrt(count + mean_background)
\end{verbatim}
\begin{center}
	\textbf{Function 5: Function used to calculate count uncertainty values (implements Equation 8)}
\end{center}
\vspace{10pt}
{\large\textbf{Data Analysis}}
\begin{verbatim}
def characterize(y: any, func: any, u: any) -> float:
     """Return the reduced chi-squared metric to determine how well a model
     function fits a given set of data using the measured data <y>, the
     prediction with the model <func> and the uncertainty on each measurement's
     dependent data <u>.
     """
     value = 0

     for i in range(np.size(y)):
          value += ((y[i] - func[i]) ** 2) / (u[i] ** 2)
          i += 1

     return value / (np.size(y) - 2)
\end{verbatim}
\begin{center}
	\textbf{Function 6: Function used to calculate chi-squared metric (implements equation 7)}
\end{center}
\vspace{5pt}
\begin{verbatim}
def count_rate(events, sample_time) -> tuple:
     """Return the count rate and its uncertainty.
     """
     return events / sample_time, np.sqrt(events) / sample_time
\end{verbatim}
\begin{center}
	\textbf{Function 7: Function used to calculate count-rate for experimental data}
\end{center}
\newpage % Probably Remove
{\Large\textbf{Sample Calculations}}
\vspace{20pt}\\
The theoretical model was derived by first combining of \textbf{Equation 4} and \textbf{Equation 5}:
$$I(t) = I_0e^{-\frac{t}{\tau}} \iff I(t) = I_0e^{\frac{t}{-\frac{t_{1/2}}{\ln{2}}}} \iff I(t) = I_0e^{-\frac{t\ln{2}}{t_{1/2}}}$$
The theoretical value for the half-life of Barium ($t_{\frac{1}{2}} = 156s$) was included in the equation:
$$I(t) = I_0e^{-\frac{t\ln{2}}{156}}$$
\begin{center}
	\textbf{Calculation 1: Deriving Theoretical Model}
\end{center}
\vspace{10pt}
All Geiger counter uncertainty calculations are based off of \textbf{Equation 8}. The mean of the background radiation count measured in the lab over each 20 second interval was $\bar{N}_b \approx 3.42$, therefore:
$$u(N_i) = \pm\sqrt{N_{total, i} + 3.42}$$
The first count value $N_1$ measured in experiment 1 was $N_1= 666$. Therefore:
$$u(N_1) = \pm\sqrt{666 + 3.42} = \pm\sqrt{669.42} \approx \pm25.87$$
Uncertainty for all values was calculated in this manner programmatically using \textbf{Function 5} which is an implementation of \textbf{Equation 8}.
\begin{center}
	\textbf{Calculation 2: Sample Geiger Counter uncertainty calculation}
\end{center}
\vspace{10pt}
The error values for the linearized plot were calculated using \textbf{Equation 6}. The first measured count value with background subtracted was $N'_{1} = 662.6$ and the uncertainty of the first measurement $u(N'_1) \approx 25.87$ as shown in \textbf{Calculation 2}. Therefore, by \textbf{Equation 6}:
$$u(\ln(N'_1))= \pm\left|\frac{u(N'_1)}{N'_1}\right|= \pm\left|\frac{25.87}{662.6}\right| \approx \pm0.039$$
\begin{center}
	\textbf{Calculation 3: Sample logarithmic error propagation}
\end{center}
\newpage
{\Large\textbf{Raw Data}}\\
\vspace{20pt}\\
\begin{center}
\begin{tabular}{| p{0.14\textwidth} | p{0.14\textwidth}|}\hline
Time Interval (s) & Total Measured Count\\
\hline
0-20 & $3$\\
20-40 & $5$\\
40-60 & $4$\\
60-80 & $2$\\
80-100 & $3$\\
100-120 & $6$\\
120-140 & $4$\\
140-160 & $2$\\
160-180 & $8$\\
180-200 & $4$\\
200-220 & $2$\\
220-240 & $2$\\
240-260 & $3$\\
260-280 & $2$\\
280-300 & $0$\\
300-320 & $8$\\
320-340 & $2$\\
340-360 & $7$\\
360-380 & $5$\\
380-400 & $3$\\
400-420 & $2$\\
420-440 & $2$\\
440-460 & $4$\\
460-480 & $3$\\
480-500 & $2$\\
500-520 & $2$\\
520-540 & $3$\\
540-560 & $3$\\
560-580 & $6$\\
580-600 & $3$\\
\hline
\end{tabular}\quad
\begin{tabular}{| p{0.14\textwidth} | p{0.14\textwidth}|}\hline
Time Interval (s) & Total Measured Count\\
\hline
600-620 & $2$\\
620-640 & $2$\\
640-660 & $0$\\
660-680 & $3$\\
680-700 & $4$\\
700-720 & $6$\\
720-740 & $4$\\
740-760 & $4$\\
760-780 & $3$\\
780-800 & $7$\\
800-820 & $3$\\
820-840 & $5$\\
840-860 & $2$\\
860-880 & $2$\\
880-900 & $3$\\
900-920 & $2$\\
920-940 & $6$\\
940-960 & $4$\\
960-980 & $2$\\
980-1000 & $9$\\
1000-1020 & $0$\\
1020-1040 & $2$\\
1040-1060 & $6$\\
1060-1080 & $3$\\
1080-1100 & $2$\\
1100-1120 & $4$\\
1120-1140 & $1$\\
1140-1160 & $7$\\
1160-1180 & $1$\\
1180-1200 & $1$\\
\hline
\end{tabular}
\end{center}
\begin{center}
	\textbf{Table 1: Raw data from the background radiation measurement, time intervals and measured counts are shown.}
\end{center}
\begin{tabular}{| p{0.14\textwidth} | p{0.14\textwidth} | p{0.14\textwidth} |}\hline
Time Interval (s) & Total Measured Count & Count without background\\
\hline
0-20 & $666$ & $663\pm 30$\\
20-40 & $628$ & $625\pm 30$\\
40-60 & $526$ & $523\pm 20$\\
60-80 & $466$ & $463\pm 20$\\
80-100 & $477$ & $474\pm 20$\\
100-120 & $382$ & $379\pm 20$\\
120-140 & $393$ & $390\pm 20$\\
140-160 & $328$ & $325\pm 20$\\
160-180 & $314$ & $311\pm 20$\\
180-200 & $266$ & $263\pm 20$\\
200-220 & $270$ & $267\pm 20$\\
220-240 & $216$ & $213\pm 10$\\
240-260 & $238$ & $235\pm 20$\\
260-280 & $181$ & $178\pm 10$\\
280-300 & $189$ & $186\pm 10$\\
300-320 & $154$ & $151\pm 10$\\
320-340 & $161$ & $158\pm 10$\\
340-360 & $142$ & $139\pm 10$\\
360-380 & $131$ & $128\pm 10$\\
380-400 & $119$ & $116\pm 10$\\
400-420 & $109$ & $106\pm 10$\\
420-440 & $96$ & $93\pm 10$\\
440-460 & $88$ & $85\pm 10$\\
460-480 & $72$ & $69\pm 9$\\
480-500 & $84$ & $81\pm 9$\\
500-520 & $63$ & $60\pm 8$\\
520-540 & $59$ & $56\pm 8$\\
540-560 & $47$ & $44\pm 7$\\
560-580 & $62$ & $59\pm 8$\\
580-600 & $54$ & $51\pm 8$\\
\hline
\end{tabular}\quad
\begin{tabular}{| p{0.14\textwidth} | p{0.14\textwidth} | p{0.14\textwidth} |}\hline
Time Interval (s) & Total Measured Count & Count without background\\
\hline
600-620 & $32$ & $29\pm 6$\\
620-640 & $32$ & $29\pm 6$\\
640-660 & $39$ & $36\pm 7$\\
660-680 & $44$ & $41\pm 7$\\
680-700 & $30$ & $27\pm 6$\\
700-720 & $38$ & $35\pm 6$\\
720-740 & $33$ & $30\pm 6$\\
740-760 & $24$ & $21\pm 5$\\
760-780 & $19$ & $16\pm 5$\\
780-800 & $24$ & $21\pm 5$\\
800-820 & $28$ & $25\pm 6$\\
820-840 & $22$ & $19\pm 5$\\
840-860 & $8$ & $5\pm 3$\\
860-880 & $18$ & $15\pm 5$\\
880-900 & $17$ & $14\pm 5$\\
900-920 & $11$ & $8\pm 4$\\
920-940 & $12$ & $9\pm 4$\\
940-960 & $15$ & $12\pm 4$\\
960-980 & $13$ & $10\pm 4$\\
980-1000 & $16$ & $13\pm 4$\\
1000-1020 & $8$ & $5\pm 3$\\
1020-1040 & $6$ & $3\pm 3$\\
1040-1060 & $3$ & $0\pm 3$\\
1060-1080 & $8$ & $5\pm 3$\\
1080-1100 & $6$ & $3\pm 3$\\
1100-1120 & $13$ & $10\pm 4$\\
1120-1140 & $8$ & $5\pm 3$\\
1140-1160 & $4$ & $1\pm 3$\\
1160-1180 & $5$ & $2\pm 3$\\
1180-1200 & $6$ & $3\pm 3$\\
\hline
\end{tabular}
\begin{center}
	\textbf{Table 2: Raw data from experiment 1 showing time intervals, total measured counts and rounded counts with background subtracted.}
\end{center}
\begin{tabular}{| p{0.14\textwidth} | p{0.14\textwidth} | p{0.14\textwidth} |}\hline
Time Interval (s) & Total Measured Count & Count without background\\
\hline
0-3 & $39$ & $36\pm 7$\\
3-6 & $39$ & $36\pm 7$\\
6-9 & $42$ & $39\pm 7$\\
9-12 & $33$ & $30\pm 6$\\
12-15 & $49$ & $46\pm 7$\\
15-18 & $32$ & $29\pm 6$\\
18-21 & $31$ & $28\pm 6$\\
21-24 & $54$ & $51\pm 8$\\
24-27 & $60$ & $57\pm 8$\\
27-30 & $40$ & $37\pm 7$\\
30-33 & $42$ & $39\pm 7$\\
33-36 & $45$ & $42\pm 7$\\
36-39 & $61$ & $58\pm 8$\\
39-42 & $45$ & $42\pm 7$\\
42-45 & $42$ & $39\pm 7$\\
45-48 & $49$ & $46\pm 7$\\
48-51 & $44$ & $41\pm 7$\\
51-54 & $41$ & $38\pm 7$\\
54-57 & $53$ & $50\pm 8$\\
57-60 & $46$ & $43\pm 7$\\
60-63 & $39$ & $36\pm 7$\\
63-66 & $35$ & $32\pm 6$\\
66-69 & $43$ & $40\pm 7$\\
69-72 & $41$ & $38\pm 7$\\
72-75 & $49$ & $46\pm 7$\\
75-78 & $50$ & $47\pm 7$\\
78-81 & $59$ & $56\pm 8$\\
81-84 & $47$ & $44\pm 7$\\
84-87 & $41$ & $38\pm 7$\\
87-90 & $37$ & $34\pm 6$\\
90-93 & $38$ & $35\pm 6$\\
93-96 & $53$ & $50\pm 8$\\
96-99 & $38$ & $35\pm 6$\\
99-102 & $37$ & $34\pm 6$\\
102-105 & $52$ & $49\pm 7$\\
105-108 & $46$ & $43\pm 7$\\
108-111 & $41$ & $38\pm 7$\\
111-114 & $32$ & $29\pm 6$\\
114-117 & $35$ & $32\pm 6$\\
117-120 & $39$ & $36\pm 7$\\
\hline
\end{tabular}\quad
\begin{tabular}{| p{0.14\textwidth} | p{0.14\textwidth} | p{0.14\textwidth} |}\hline
Time Interval (s) & Total Measured Count & Count without background\\
\hline
120-123 & $39$ & $36\pm 7$\\
123-126 & $39$ & $36\pm 7$\\
126-129 & $34$ & $31\pm 6$\\
129-132 & $46$ & $43\pm 7$\\
132-135 & $42$ & $39\pm 7$\\
135-138 & $39$ & $36\pm 7$\\
138-141 & $52$ & $49\pm 7$\\
141-144 & $51$ & $48\pm 7$\\
144-147 & $59$ & $56\pm 8$\\
147-150 & $42$ & $39\pm 7$\\
150-153 & $52$ & $49\pm 7$\\
153-156 & $36$ & $33\pm 6$\\
156-159 & $48$ & $45\pm 7$\\
159-162 & $47$ & $44\pm 7$\\
162-165 & $44$ & $41\pm 7$\\
165-168 & $41$ & $38\pm 7$\\
168-171 & $37$ & $34\pm 6$\\
171-174 & $53$ & $50\pm 8$\\
174-177 & $45$ & $42\pm 7$\\
177-180 & $52$ & $49\pm 7$\\
180-183 & $44$ & $41\pm 7$\\
183-186 & $42$ & $39\pm 7$\\
186-189 & $53$ & $50\pm 8$\\
189-192 & $47$ & $44\pm 7$\\
192-195 & $51$ & $48\pm 7$\\
195-198 & $45$ & $42\pm 7$\\
198-201 & $41$ & $38\pm 7$\\
201-204 & $46$ & $43\pm 7$\\
204-207 & $39$ & $36\pm 7$\\
207-210 & $54$ & $51\pm 8$\\
210-213 & $51$ & $48\pm 7$\\
213-216 & $45$ & $42\pm 7$\\
216-219 & $35$ & $32\pm 6$\\
219-222 & $44$ & $41\pm 7$\\
222-225 & $46$ & $43\pm 7$\\
225-228 & $40$ & $37\pm 7$\\
228-231 & $44$ & $41\pm 7$\\
231-234 & $47$ & $44\pm 7$\\
234-237 & $37$ & $34\pm 6$\\
237-240 & $36$ & $33\pm 6$\\
\hline
\end{tabular}\\
\begin{tabular}{| p{0.14\textwidth} | p{0.14\textwidth} | p{0.14\textwidth} |}\hline
Time Interval (s) & Total Measured Count & Count without background\\
\hline
240-243 & $40$ & $37\pm 7$\\
243-246 & $32$ & $29\pm 6$\\
246-249 & $51$ & $48\pm 7$\\
249-252 & $40$ & $37\pm 7$\\
252-255 & $33$ & $30\pm 6$\\
255-258 & $45$ & $42\pm 7$\\
258-261 & $37$ & $34\pm 6$\\
261-264 & $58$ & $55\pm 8$\\
264-267 & $50$ & $47\pm 7$\\
267-270 & $37$ & $34\pm 6$\\
270-273 & $38$ & $35\pm 6$\\
273-276 & $48$ & $45\pm 7$\\
276-279 & $49$ & $46\pm 7$\\
279-282 & $39$ & $36\pm 7$\\
282-285 & $40$ & $37\pm 7$\\
285-288 & $45$ & $42\pm 7$\\
288-291 & $43$ & $40\pm 7$\\
291-294 & $55$ & $52\pm 8$\\
294-297 & $37$ & $34\pm 6$\\
297-300 & $36$ & $33\pm 6$\\
300-303 & $40$ & $37\pm 7$\\
303-306 & $40$ & $37\pm 7$\\
306-309 & $38$ & $35\pm 6$\\
309-312 & $43$ & $40\pm 7$\\
312-315 & $51$ & $48\pm 7$\\
315-318 & $49$ & $46\pm 7$\\
318-321 & $58$ & $55\pm 8$\\
321-324 & $36$ & $33\pm 6$\\
324-327 & $47$ & $44\pm 7$\\
327-330 & $41$ & $38\pm 7$\\
330-333 & $43$ & $40\pm 7$\\
333-336 & $48$ & $45\pm 7$\\
336-339 & $42$ & $39\pm 7$\\
339-342 & $50$ & $47\pm 7$\\
342-345 & $53$ & $50\pm 8$\\
345-348 & $49$ & $46\pm 7$\\
348-351 & $51$ & $48\pm 7$\\
351-354 & $41$ & $38\pm 7$\\
354-357 & $45$ & $42\pm 7$\\
357-360 & $37$ & $34\pm 6$\\
\hline
\end{tabular}\quad
\begin{tabular}{| p{0.14\textwidth} | p{0.14\textwidth} | p{0.14\textwidth} |}\hline
Time Interval (s) & Total Measured Count & Count without background\\
\hline
360-363 & $30$ & $27\pm 6$\\
363-366 & $40$ & $37\pm 7$\\
366-369 & $44$ & $41\pm 7$\\
369-372 & $47$ & $44\pm 7$\\
372-375 & $48$ & $45\pm 7$\\
375-378 & $41$ & $38\pm 7$\\
378-381 & $34$ & $31\pm 6$\\
381-384 & $41$ & $38\pm 7$\\
384-387 & $41$ & $38\pm 7$\\
387-390 & $43$ & $40\pm 7$\\
390-393 & $47$ & $44\pm 7$\\
393-396 & $45$ & $42\pm 7$\\
396-399 & $39$ & $36\pm 7$\\
399-402 & $47$ & $44\pm 7$\\
402-405 & $45$ & $42\pm 7$\\
405-408 & $27$ & $24\pm 6$\\
408-411 & $50$ & $47\pm 7$\\
411-414 & $47$ & $44\pm 7$\\
414-417 & $44$ & $41\pm 7$\\
417-420 & $47$ & $44\pm 7$\\
420-423 & $46$ & $43\pm 7$\\
423-426 & $48$ & $45\pm 7$\\
426-429 & $33$ & $30\pm 6$\\
429-432 & $34$ & $31\pm 6$\\
432-435 & $42$ & $39\pm 7$\\
435-438 & $43$ & $40\pm 7$\\
438-441 & $39$ & $36\pm 7$\\
441-444 & $46$ & $43\pm 7$\\
444-447 & $38$ & $35\pm 6$\\
447-450 & $40$ & $37\pm 7$\\
450-453 & $36$ & $33\pm 6$\\
453-456 & $49$ & $46\pm 7$\\
456-459 & $50$ & $47\pm 7$\\
459-462 & $40$ & $37\pm 7$\\
462-465 & $50$ & $47\pm 7$\\
465-468 & $53$ & $50\pm 8$\\
468-471 & $54$ & $51\pm 8$\\
471-474 & $37$ & $34\pm 6$\\
474-477 & $41$ & $38\pm 7$\\
477-480 & $44$ & $41\pm 7$\\
\hline
\end{tabular}\\
\begin{tabular}{| p{0.14\textwidth} | p{0.14\textwidth} | p{0.14\textwidth} |}\hline
Time Interval (s) & Total Measured Count & Count without background\\
\hline
480-483 & $58$ & $55\pm 8$\\
483-486 & $42$ & $39\pm 7$\\
486-489 & $51$ & $48\pm 7$\\
489-492 & $41$ & $38\pm 7$\\
492-495 & $50$ & $47\pm 7$\\
495-498 & $45$ & $42\pm 7$\\
498-501 & $41$ & $38\pm 7$\\
501-504 & $42$ & $39\pm 7$\\
504-507 & $48$ & $45\pm 7$\\
507-510 & $42$ & $39\pm 7$\\
510-513 & $39$ & $36\pm 7$\\
513-516 & $50$ & $47\pm 7$\\
516-519 & $47$ & $44\pm 7$\\
519-522 & $49$ & $46\pm 7$\\
522-525 & $47$ & $44\pm 7$\\
525-528 & $33$ & $30\pm 6$\\
528-531 & $37$ & $34\pm 6$\\
531-534 & $52$ & $49\pm 7$\\
534-537 & $46$ & $43\pm 7$\\
537-540 & $38$ & $35\pm 6$\\
540-543 & $42$ & $39\pm 7$\\
543-546 & $37$ & $34\pm 6$\\
546-549 & $30$ & $27\pm 6$\\
549-552 & $48$ & $45\pm 7$\\
552-555 & $47$ & $44\pm 7$\\
555-558 & $38$ & $35\pm 6$\\
558-561 & $41$ & $38\pm 7$\\
561-564 & $34$ & $31\pm 6$\\
564-567 & $35$ & $32\pm 6$\\
567-570 & $36$ & $33\pm 6$\\
570-573 & $39$ & $36\pm 7$\\
573-576 & $36$ & $33\pm 6$\\
576-579 & $43$ & $40\pm 7$\\
579-582 & $55$ & $52\pm 8$\\
582-585 & $43$ & $40\pm 7$\\
585-588 & $45$ & $42\pm 7$\\
588-591 & $32$ & $29\pm 6$\\
591-594 & $51$ & $48\pm 7$\\
594-597 & $45$ & $42\pm 7$\\
597-600 & $57$ & $54\pm 8$\\
\hline
\end{tabular}\quad
\begin{tabular}{| p{0.14\textwidth} | p{0.14\textwidth} | p{0.14\textwidth} |}\hline
Time Interval (s) & Total Measured Count & Count without background\\
\hline
600-603 & $32$ & $29\pm 6$\\
603-606 & $48$ & $45\pm 7$\\
606-609 & $47$ & $44\pm 7$\\
609-612 & $42$ & $39\pm 7$\\
612-615 & $24$ & $21\pm 5$\\
615-618 & $55$ & $52\pm 8$\\
618-621 & $54$ & $51\pm 8$\\
621-624 & $43$ & $40\pm 7$\\
624-627 & $48$ & $45\pm 7$\\
627-630 & $43$ & $40\pm 7$\\
630-633 & $59$ & $56\pm 8$\\
633-636 & $33$ & $30\pm 6$\\
636-639 & $41$ & $38\pm 7$\\
639-642 & $45$ & $42\pm 7$\\
642-645 & $54$ & $51\pm 8$\\
645-648 & $36$ & $33\pm 6$\\
648-651 & $40$ & $37\pm 7$\\
651-654 & $41$ & $38\pm 7$\\
654-657 & $54$ & $51\pm 8$\\
657-660 & $42$ & $39\pm 7$\\
660-663 & $42$ & $39\pm 7$\\
663-666 & $46$ & $43\pm 7$\\
666-669 & $43$ & $40\pm 7$\\
669-672 & $30$ & $27\pm 6$\\
672-675 & $46$ & $43\pm 7$\\
675-678 & $47$ & $44\pm 7$\\
678-681 & $39$ & $36\pm 7$\\
681-684 & $55$ & $52\pm 8$\\
684-687 & $45$ & $42\pm 7$\\
687-690 & $47$ & $44\pm 7$\\
690-693 & $46$ & $43\pm 7$\\
693-696 & $50$ & $47\pm 7$\\
696-699 & $40$ & $37\pm 7$\\
699-702 & $44$ & $41\pm 7$\\
702-705 & $66$ & $63\pm 8$\\
705-708 & $39$ & $36\pm 7$\\
708-711 & $43$ & $40\pm 7$\\
711-714 & $42$ & $39\pm 7$\\
714-717 & $55$ & $52\pm 8$\\
717-720 & $41$ & $38\pm 7$\\
\hline
\end{tabular}\\
\begin{tabular}{| p{0.14\textwidth} | p{0.14\textwidth} | p{0.14\textwidth} |}\hline
Time Interval (s) & Total Measured Count & Count without background\\
\hline
720-723 & $49$ & $46\pm 7$\\
723-726 & $58$ & $55\pm 8$\\
726-729 & $45$ & $42\pm 7$\\
729-732 & $47$ & $44\pm 7$\\
732-735 & $42$ & $39\pm 7$\\
735-738 & $61$ & $58\pm 8$\\
738-741 & $44$ & $41\pm 7$\\
741-744 & $35$ & $32\pm 6$\\
744-747 & $39$ & $36\pm 7$\\
747-750 & $44$ & $41\pm 7$\\
750-753 & $35$ & $32\pm 6$\\
753-756 & $41$ & $38\pm 7$\\
756-759 & $53$ & $50\pm 8$\\
759-762 & $47$ & $44\pm 7$\\
762-765 & $63$ & $60\pm 8$\\
765-768 & $35$ & $32\pm 6$\\
768-771 & $47$ & $44\pm 7$\\
771-774 & $36$ & $33\pm 6$\\
774-777 & $37$ & $34\pm 6$\\
777-780 & $33$ & $30\pm 6$\\
780-783 & $43$ & $40\pm 7$\\
783-786 & $33$ & $30\pm 6$\\
786-789 & $44$ & $41\pm 7$\\
789-792 & $44$ & $41\pm 7$\\
792-795 & $39$ & $36\pm 7$\\
795-798 & $41$ & $38\pm 7$\\
798-801 & $48$ & $45\pm 7$\\
801-804 & $34$ & $31\pm 6$\\
804-807 & $36$ & $33\pm 6$\\
807-810 & $40$ & $37\pm 7$\\
810-813 & $41$ & $38\pm 7$\\
813-816 & $38$ & $35\pm 6$\\
816-819 & $43$ & $40\pm 7$\\
819-822 & $46$ & $43\pm 7$\\
822-825 & $39$ & $36\pm 7$\\
825-828 & $37$ & $34\pm 6$\\
828-831 & $44$ & $41\pm 7$\\
831-834 & $47$ & $44\pm 7$\\
834-837 & $46$ & $43\pm 7$\\
837-840 & $42$ & $39\pm 7$\\
\hline
\end{tabular}\quad
\begin{tabular}{| p{0.14\textwidth} | p{0.14\textwidth} | p{0.14\textwidth} |}\hline
Time Interval (s) & Total Measured Count & Count without background\\
\hline
840-843 & $36$ & $33\pm 6$\\
843-846 & $34$ & $31\pm 6$\\
846-849 & $50$ & $47\pm 7$\\
849-852 & $53$ & $50\pm 8$\\
852-855 & $24$ & $21\pm 5$\\
855-858 & $48$ & $45\pm 7$\\
858-861 & $49$ & $46\pm 7$\\
861-864 & $45$ & $42\pm 7$\\
864-867 & $50$ & $47\pm 7$\\
867-870 & $27$ & $24\pm 6$\\
870-873 & $33$ & $30\pm 6$\\
873-876 & $47$ & $44\pm 7$\\
876-879 & $37$ & $34\pm 6$\\
879-882 & $41$ & $38\pm 7$\\
882-885 & $45$ & $42\pm 7$\\
885-888 & $41$ & $38\pm 7$\\
888-891 & $49$ & $46\pm 7$\\
891-894 & $31$ & $28\pm 6$\\
894-897 & $41$ & $38\pm 7$\\
897-900 & $51$ & $48\pm 7$\\
900-903 & $36$ & $33\pm 6$\\
903-906 & $43$ & $40\pm 7$\\
906-909 & $48$ & $45\pm 7$\\
909-912 & $43$ & $40\pm 7$\\
912-915 & $46$ & $43\pm 7$\\
915-918 & $40$ & $37\pm 7$\\
918-921 & $48$ & $45\pm 7$\\
921-924 & $37$ & $34\pm 6$\\
924-927 & $40$ & $37\pm 7$\\
927-930 & $46$ & $43\pm 7$\\
930-933 & $44$ & $41\pm 7$\\
933-936 & $39$ & $36\pm 7$\\
936-939 & $45$ & $42\pm 7$\\
939-942 & $39$ & $36\pm 7$\\
942-945 & $40$ & $37\pm 7$\\
945-948 & $40$ & $37\pm 7$\\
948-951 & $39$ & $36\pm 7$\\
951-954 & $44$ & $41\pm 7$\\
954-957 & $45$ & $42\pm 7$\\
957-960 & $37$ & $34\pm 6$\\
\hline
\end{tabular}\\
\begin{tabular}{| p{0.14\textwidth} | p{0.14\textwidth} | p{0.14\textwidth} |}\hline
Time Interval (s) & Total Measured Count & Count without background\\
\hline
960-963 & $44$ & $41\pm 7$\\
963-966 & $40$ & $37\pm 7$\\
966-969 & $41$ & $38\pm 7$\\
969-972 & $54$ & $51\pm 8$\\
972-975 & $37$ & $34\pm 6$\\
975-978 & $35$ & $32\pm 6$\\
978-981 & $50$ & $47\pm 7$\\
981-984 & $39$ & $36\pm 7$\\
984-987 & $38$ & $35\pm 6$\\
987-990 & $38$ & $35\pm 6$\\
990-993 & $53$ & $50\pm 8$\\
993-996 & $43$ & $40\pm 7$\\
996-999 & $49$ & $46\pm 7$\\
999-1002 & $33$ & $30\pm 6$\\
1002-1005 & $39$ & $36\pm 7$\\
1005-1008 & $40$ & $37\pm 7$\\
1008-1011 & $49$ & $46\pm 7$\\
1011-1014 & $45$ & $42\pm 7$\\
1014-1017 & $40$ & $37\pm 7$\\
1017-1020 & $35$ & $32\pm 6$\\
1020-1023 & $46$ & $43\pm 7$\\
1023-1026 & $44$ & $41\pm 7$\\
1026-1029 & $40$ & $37\pm 7$\\
1029-1032 & $50$ & $47\pm 7$\\
1032-1035 & $42$ & $39\pm 7$\\
1035-1038 & $35$ & $32\pm 6$\\
1038-1041 & $32$ & $29\pm 6$\\
1041-1044 & $51$ & $48\pm 7$\\
1044-1047 & $58$ & $55\pm 8$\\
1047-1050 & $46$ & $43\pm 7$\\
1050-1053 & $47$ & $44\pm 7$\\
1053-1056 & $47$ & $44\pm 7$\\
1056-1059 & $42$ & $39\pm 7$\\
1059-1062 & $44$ & $41\pm 7$\\
1062-1065 & $43$ & $40\pm 7$\\
1065-1068 & $39$ & $36\pm 7$\\
1068-1071 & $42$ & $39\pm 7$\\
1071-1074 & $45$ & $42\pm 7$\\
1074-1077 & $36$ & $33\pm 6$\\
1077-1080 & $40$ & $37\pm 7$\\
\hline
\end{tabular}\quad
\begin{tabular}{| p{0.14\textwidth} | p{0.14\textwidth} | p{0.14\textwidth} |}\hline
Time Interval (s) & Total Measured Count & Count without background\\
\hline
1080-1083 & $38$ & $35\pm 6$\\
1083-1086 & $42$ & $39\pm 7$\\
1086-1089 & $35$ & $32\pm 6$\\
1089-1092 & $38$ & $35\pm 6$\\
1092-1095 & $39$ & $36\pm 7$\\
1095-1098 & $49$ & $46\pm 7$\\
1098-1101 & $51$ & $48\pm 7$\\
1101-1104 & $44$ & $41\pm 7$\\
1104-1107 & $38$ & $35\pm 6$\\
1107-1110 & $39$ & $36\pm 7$\\
1110-1113 & $44$ & $41\pm 7$\\
1113-1116 & $43$ & $40\pm 7$\\
1116-1119 & $31$ & $28\pm 6$\\
1119-1122 & $52$ & $49\pm 7$\\
1122-1125 & $48$ & $45\pm 7$\\
1125-1128 & $43$ & $40\pm 7$\\
1128-1131 & $35$ & $32\pm 6$\\
1131-1134 & $51$ & $48\pm 7$\\
1134-1137 & $44$ & $41\pm 7$\\
1137-1140 & $44$ & $41\pm 7$\\
1140-1143 & $46$ & $43\pm 7$\\
1143-1146 & $38$ & $35\pm 6$\\
1146-1149 & $48$ & $45\pm 7$\\
1149-1152 & $46$ & $43\pm 7$\\
1152-1155 & $42$ & $39\pm 7$\\
1155-1158 & $52$ & $49\pm 7$\\
1158-1161 & $40$ & $37\pm 7$\\
1161-1164 & $46$ & $43\pm 7$\\
1164-1167 & $41$ & $38\pm 7$\\
1167-1170 & $44$ & $41\pm 7$\\
1170-1173 & $40$ & $37\pm 7$\\
1173-1176 & $45$ & $42\pm 7$\\
1176-1179 & $44$ & $41\pm 7$\\
1179-1182 & $33$ & $30\pm 6$\\
1182-1185 & $40$ & $37\pm 7$\\
1185-1188 & $42$ & $39\pm 7$\\
1188-1191 & $45$ & $42\pm 7$\\
1191-1194 & $33$ & $30\pm 6$\\
1194-1197 & $34$ & $31\pm 6$\\
1197-1200 & $48$ & $45\pm 7$\\
\hline
\end{tabular}
\begin{center}
	\textbf{Table 3: Raw data from experiment 2 showing time intervals, total measured counts and rounded counts with background subtracted.}
\end{center}
\end{document}